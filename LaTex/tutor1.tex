\documentclass[11pt]{article}

\begin{document}
This is my first LaTex documents.

I'm doing great now!

Area of a Square:

find the area of square with lengths $(x+6)$.

Answer:

The area of a square is given by


length $.$ length, $length^2$

$Area = (x+6)^2$. this represents the area of the square.


Area of a Square:

find the area of square with lengths $$(x+6)$$

Answer:

The area of a square is given by


length $.$ length, $$length^2$$

$$Area = (x+6)^2$$this represents the area of the square.


Writing superScripts: $$4x^8$$

$$45x^{4x+5}$$
$$7x^{5x^{32}+56} + 9x^t + 44$$



writing subScripts: $$x_0 + x_1 + x_3$$
$$x_{i2}+x_{i3}$$


greek letters:
$$\pi r^2$$
$$\alpha$$
$$\beta$$


Trig functions:
$$y=\sin(x^2)$$
$$h=\tan{x}$$


log functions:$$\log_5{x}$$
$$\ln{3x}$$


Sqaure roots:
$$\sqrt[4]{56}$$
$$\sqrt{x^3+34x+12}$$
$$\varsigma$$
$$\beta \iint \mathbb{P}$$
$$M \varLambda $$

$$\tau (\omega) = inf\{n\,\epsilon \,\mathbb{I} :\mathsf{X}_n(\omega)< 10\} $$

1. Find \mathbb{F}^\mathsf{X}  = \{\mathcal{F}_n^X:n\,\epsilon \,\mathbb{I} \}  = \{\mathcal{F}_0^\mathsf{X} , \mathcal{F}_1^\mathsf{X},  \mathcal{F}_2^\mathsf{X}  \}


\begin{eqnarray}

\end{eqnarray}

\end{array}
\\
\\
The natural filtration of X are these three $\sigma$ algebras which are subset $\sigma$-algebras of $\mathbb{F}$ which is the powerset of $\Omega$







\end{document}